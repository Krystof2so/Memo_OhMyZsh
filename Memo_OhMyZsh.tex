\documentclass[a4paper,11pt]{article}

\usepackage[english,french]{babel}
\usepackage[utf8]{inputenc}
\usepackage[T1]{fontenc}
\usepackage{textcomp}
\usepackage{color}
\usepackage{hyperref}

\title{MEMO Oh-My-Zsh}
\author{krystof}
\date{\today}

\begin{document}
\maketitle

\textbf{Oh-My-Zsh} propose un cadre pour la gestion de la configuration de \textbf{zsh}. Il utilise pour cela de multiples fonctions, des plugins et des thèmes personnalisés et utiles.
\medskip

\section{Installation}
Lors de son installation \textbf{Oh-My-Zsh} crée une sauvegarde du fichier \textbf{.zshrc} puis le remplace par sa propre version. Toute configuration personnalisée doit alors être copiée dans ce fichier.
\medskip

\section{Sélectionner un thème}
\begin{itemize}
	\item Via la variable \verb|ZSH_THEME| dans le fichier \textbf{.zshrc}.
	\item Liste des thèmes livrés avec \textbf{Oh-My-Zsh}: \url{https://github.com/ohmyzsh/tree/master/themes}.
	\item Activer un thème: \verb|ZSH_THEME=nomduthème|.
	\item Ne pas activer un thème: \verb|ZSH_THEME=""|.
	\item Captures d'écran et descriptions de thèmes: \url{https://github.com/ohmyzsh/ohmyzsh/wiki/Themes}.
	\item D'autres thèmes: 
	
	\url{https://github.com/ohmyzsh/ohmyzsh/wiki/External-themes}.
\end{itemize}
\medskip

\section{Les plugins}
Ajouter un plugin: \verb|plugins=(nom_plugin)| dans le fichier \textbf{.zshrc}. Pour une liste de plugins, voir:

\url{https://github.com/ohmyzsh/ohmyzsh/wiki/plugins}
\medskip

\section{L'auto-suggestion \textbf{FTW!}}
Un plugin d'auto-suggestion suggère des commandes au fur et à mesure de leur saisie en se basant sur l'historique.

Installation par le clonage du dépôt:
\begin{verbatim}
$ git clone https://github.com/zsh-users/zsh-autosuggestions ${
ZSH_CUSTOM:~/.oh-my-zsh/custom}/plugins/zsh-autosuggestions|
\end{verbatim}

Activation: \verb|plugins=(zsh-autosuggestions)|.

Si la suggestion proposée est celle recherchée, appuyer alors sur $\rightarrow$ pour l'accepter.
\medskip

\section{Navigation dans les fichiers}
Pour naviguer dans les répertoires du système de fichier il n'est pas nécessaire d'utiliser la commande \verb|cd|. Commencer par saisir le nom d'un chemin, puis appuyer sur la touche \textbf{<Tab>} pour dérouler une liste de tous les répertoires possibles. Continuer d'appuyer sur la touche \textbf{<Tab>} pour sélectionner un répertoire et taper \textbf{<Entrée>} pour se rendre dans le répertoire.
\medskip

Pour une liste des alias liés à la navigation dans les répertoires:

\url{https://github.com/ohmyzsh/ohmyzsh/wiki/Cheatsheet#directory}
\medskip

\textbf{Oh-My-Zsh} supporte la complétion de chemin dynamique. Par exemple, en saisissant: \verb|$ ~/Do/DE/| suivi de la touche \textbf{<Tab>}, le chemin sera alors étendu au répertoire \verb|~/Documents/DEVELOPPEMENT|.
\medskip

Il est même possible de faire glisser un répertoire dans la fenêtre du terminal pour accéder au répertoire dans le terminal.
\medskip

\section{Travailler avec \textbf{git}}
Avec le plugins \textbf{git} nous disposons de toute une kyrielle d'alias. Voir:

\url{https://github.com/ohmyzsh/ohmyzsh/tree/master/plugins/git}
\medskip

Possibilité de créer son propre alias réunissant plusieurs commandes. Par exemple, ajouter les lignes suivantes dans \textbf{.zshrc}:
\begin{verbatim}
	function acp(){
		git add.
		git commit -m "$1"
		git push
	}
\end{verbatim}
Saisir alors \verb|$ acp "mon message de commit"|: ajoute tous les fichiers modifiés, réalise un commit avec le message saisi, puis envoie le tout vers le \textit{repository}.

\end{document}
